\input cwebmac

\font\eightit=cmti8
\font\tenss=cmss10
\font\tenssi=cmssi10
\font\tenssb=cmssdc10
\let\cmntfont=\tenssi
\def\mainfont{\def\rm{\fam0\tenrm}%
  \def\it{\fam\itfam\tenit}%
  \def\bf{\fam\bffam\tenbf}%
  \normalbaselines\rm}
\def\codefont{\def\rm{\fam0\tenss}%
  \def\it{\fam\itfam\tenssi}%
  \def\bf{\fam\bffam\tenssb}%
  \normalbaselines\rm}
\mainfont

\let\1=\quad
\let\2=\qquad

{\catcode`\(=\active \catcode`\)=\active
 \gdef({{\rm\char`\(}\kern.025em}
 \gdef){\/{\rm\char`\)}\kern.025em}
 \gdef\activeparens{\catcode`\(=\active \catcode`\)=\active}}
{\catcode`\*=\active
 \gdef\activestar{\catcode`\*=\active \def*{\leavevmode$\ast$}}}
{\obeyspaces\global\let =\ } % let active space = control space

\def\:#1{\hbox{\it:\/#1\/\kern.05em}} % keyword
\def\'{'\kern.025em} % quote
\def\`{`\kern.025em} % backquote
\def\#{\char`\#}
\def\&{\char`\&}
\def\B{\bgroup % go into Lisp mode
  \rightskip=0pt plus 100pt minus 10pt
  \parindent=0pt
  \pretolerance 10000
  \hyphenpenalty 1000 % so strings can be broken (discretionary \ is inserted)
  \exhyphenpenalty 10000
  \let\!=\cleartabs
  \IB}
\def\IB{\activeparens\activestar\codefont\obeyspaces} % (inner-)Lisp mode
\def\({\bgroup\IB\it} % go into inner-Lisp mode
\def\){\/\egroup} % end inner-Lisp mode
\def\C#1{{\cmntfont #1}} % comment
\def\CH#1{\leavevmode$\backslash$\.{#1}} % character
\def\CO#1{\leavevmode,#1\kern.025em} % comma(-at, -dot)
\def\L{\leavevmode$\lambda$}
\def\K#1{\hbox{\it#1\/\kern.05em}} % lambda-list keyword
\def\RC#1{{\it #1}} % read-time conditional
\def\T{\exnote{This test exercises code in or around {\it\progname\/} section}}
\def\Ts{\exnote{These tests exercise code in or around {\it\progname\/} section}}
\def\X#1:#2\X{\ifmmode\gdef\XX{\null$\null}\else\gdef\XX{}\fi %$% section name
  \XX$\langle\,${\mainfont\let\I=\ne#2\eightrm\kern.5em
    \ifacro{\pdfnote#1.}\else#1\fi}$\,\rangle$\XX}

% We don't want to actually make hyperlinks between the test notes and the
% main program sections.
\def\nolink#1#2{#1}
\def\exnote#1#2.{{\let\pdflink\nolink \let\it\eightit \note{#1}{#2}.}}
